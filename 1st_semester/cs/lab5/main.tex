\documentclass{article}
\usepackage[english, russian]{babel}
\usepackage{mathtools}
\begin{document}

\twocolumn

\emph{Каково должно быть значение a, чтобы через 3 часа после возобновления движения автомобиль находился ближе всего к пункту B?}

Пусть автомобиль остановился в точке С, а через 3 часа был в точке D.
Рассмотрим движение автомобиля на участке \emph{BC}. По условию скорость автомобиля в пункте \emph{B} равна 
\begin{math}
    v_{B} = 42
\end{math} \emph{км/ч}; эта скорость является начальной на участке \emph{BC}. Конечная скорость является на рассматриваемом участке 
\begin{math}
    v_{c} = 0 
\end{math}, поэтому время движения на участке \emph{BC} (в часах) \begin{math}
 t_{BC} = \frac{42}{a}   
\end{math}.

Используя формулу длины пути при равнопеременном прямолинейном движении, выразим \emph{BC} и \emph{CD}


и \emph{CD} через параметра \emph{a}:

\begin{ceqn}
\begin{align}
BC = \frac{882}{a}, CD=\frac{9a}{2}
\end{align}
\end{ceqn}


\end{document}
