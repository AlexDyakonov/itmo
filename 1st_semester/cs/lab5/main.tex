\documentclass{article}
\usepackage[english, russian]{babel}
\usepackage{mathtools}
\usepackage{multicol}
\begin{document}

\begin{math}
    \tau(\emph{p}) = 2
\end{math}, и простых чисел бесконечно много, т.е как угодно далеко во второй строке нашей таблицы будут попадаться двойки; с другой стороны, ясно, что для некоторых \emph{N} число делителей может быть сколь угодно большим -- чтобы добиться этого, нужно лишь взять число, в каноническом разложении которого показатели 
\begin{math}
\alpha_1, \alpha_2 ... \alpha_k
\end{math} достаточно велики.
На рисунке 1 изображен график функции \begin{math}
    \tau(N)
\end{math}, точки для наглядности соединены отрезками. Видите, какие получаются "горы и ущелья"?

От точной формулы толку мало -- слишком уж нерегулярна наша функция. Нет ли более наглядной, пусть приблизительной, формулы, которая бы сразу показывала, чего ожидать от \begin{math}
    \tau(N)
\end{math}

Посмотрим, как ведет себя \begin{math}
    \tau(N)
\end{math} "в среднем". Возьмем среднее арифмитическое числа делителей первых \emph{N} натуральных чисел и обозначим его через \begin{math}
    \tau_{cp}(N):
\end{math} 

\begin{math}
     \tau_{cp}(N) = \frac{1}{n}(\tau(1) + \tau(2) + ... + \tau(N))
\end{math}

Оказывается, для такой функции есть хорошая формула. Она не абсолютна точна, зато выражает "среднее число делителей" через хорошую известную функцию:

\begin{math}
    \tau_{cp}(N) = \ln N
\end{math}.

\end{document}