\documentclass{article}
\usepackage[english, russian]{babel}
\usepackage[margin={2cm,2cm}]{geometry}
\usepackage{mathtools}
\usepackage{graphicx}
\usepackage{multicol}
\graphicspath{./images}

\begin{document}
\small
\begin{multicols*}{3}
\noindent$\tau(\emph{p}) = 2$, и простых чисел бесконечно много, т.е как угодно далеко во второй строке нашей таблицы будут попадаться двойки; с другой стороны, ясно, что для некоторых \emph{N} число делителей может быть сколь угодно большим -- чтобы добиться этого, нужно лишь взять число, в каноническом разложении которого показатели 
\begin{math}
\alpha_1, \alpha_2 ... \alpha_k
\end{math} достаточно велики.
На рисунке 1 изображен график функции \begin{math}
    \tau(N)
\end{math}, точки для наглядности соединены отрезками. Видите, какие получаются "горы и ущелья"?

От точной формулы толку мало -- слишком уж нерегулярна наша функция. Нет ли более наглядной, пусть приблизительной, формулы, которая бы сразу показывала, чего ожидать от \begin{math}
    \tau(N)
\end{math}

Посмотрим, как ведет себя \begin{math}
    \tau(N)
\end{math} "в среднем". Возьмем среднее арифмитическое числа делителей первых \emph{N} натуральных чисел и обозначим его через \begin{math}
    \tau_{cp}(N):
\end{math} 

\begin{math}
     \tau_{cp}(N) = \frac{1}{n}(\tau(1) + \tau(2) + ... + \tau(N))
\end{math}

Оказывается, для такой функции есть хорошая формула. Она не абсолютно

\noindent\includegraphics[scale=0.4]{images/plot1.png}
\includegraphics[scale=0.4]{images/plot2.png}

\vspace{30px}

\noindentточна, зато выражает "среднее число делителей" через хорошую известную функцию:

\begin{math}
    \tau_{cp}(N) = \ln N
\end{math}.

\noindent\rule{5.5cm}{0.5pt}

\noindent\large{\textbf{Причем здесь логарифм?}}
\noindent\rule{5.5cm}{0.5pt}

\small
Откуда взялся логарифм? На первый взгляд его появление выглядит довольно странно. Но на самом деле ничего удивительного здесь нет. Например, для \begin{math}
    N = 2^n
\end{math}

\noindent
\small{
\begin{math}
   \tau(N) = \tau(2^n) = n + 1 = \log_2 N + 1 = 
\end{math}
}

\begin{math}
 = \log_2 N + \log_2 = \log_2(2N)
\end{math}

Конечно, пример не очень показательный, поскольку такое натуральное число - степень простого - явление редкое, да и логарифм получился не натуральный, а по основанию 2.

Мы докажем справедливость предложенной формулы чуть ниже, но сначала немного уточним ее. Что означает здесь приближенное равенство? Существует число \begin{math}
    \mu
\end{math}, приблизительно равное 0,154, такое что
\small{
\begin{math}
    \tau_{cp}(N) = \ln N + \mu + a_N
\end{math}
}

\noindent причем \begin{math} a_N \end{math} -- бесконечно малая последовательность, т.е при стремлении N к бесконечности ее предел становится равным нулю. При больших номерах N число \begin{math}
    a_N
\end{math} становится сколь угодно малым, и им можно пренебречь по сравнению с постоянным числом \begin{math}\mu\end{math} и уж тем более по сравнению с растущей функцией \begin{math}\ln N\end{math}. 
Вот в чем смысл приближенного равенства $\tau(N) \approx \ln N$.

\noindent\rule{5.5cm}{0.5pt}

\noindent\large{\textbf{Среднее число делителей}}
\noindent\rule{5.5cm}{0.5pt}

\small
Сначала научимся <<видеть>> делители натурального числа. Например, выпишем все делители числа 12:

\includegraphics{images/plot3.png}
\vspace{10px}

\begin{center}
    1, 2, 3, 4, 6, 12
\end{center}

\noindentРассмотрим функцию $f(x) = 12/x$. Ее график -- всем известная гипербола. Нам понадобится только одна ее ветвь -- та, что расположена в первом квадранте. Построим график: вычислим значения функции для заданных значений аргумента $x$: при $x$ = 1, 2, 3, ... получим $y$ = 12, 6, 4, ... Удобно вычислять $y$ для значений $x$, совпадающих с делителями числа 12. Полученные точки с целочисленными координатами отметим на координатной плоскости и соединим их кривой (рис. 2).

Посмотрим теперь, сколько точек с целочисленным координатами (их называют <<целыми точками>>; на рисунке 3 отмечены все целые точки в окрестности начала координат) попали на построенную нами ветвь гиперболы. Их ровно 6 -- столько же, сколько делителей у числа 12. Ведь каждому натуральному делителю $x$ соответствует натуральное число $y$, такое что $xy =12$.

На рисунке 4 вместе с ветвью гиперболы $y = 12/x$ изображена ветвь гиперболы $y = 6/x$; на ней столько же целых точек, сколько делитетелей у числа 6. Но ведь каждая целая точка в первом квадранте под ветвью гиперболы $y = 12/x$, не считая точек, лежащих на координатных осях, лежит на какой-то гиперболе $y = n/x$, причем $n<12$. Например, через точку (1, 11) проходит гипербола $y = 11/x$, а через точку (2, 2) -- гипербола $y = 4/x$.

Выходит, число делителей у всех натуральных чисел, меньших либо равных 12, -- это число целых точек, лежащих в первом квадранте под гиперболой $y = 12/x$ и на самой этой гиперболе  (не считая точек, лежащих на координатных осях).

\includegraphics[scale=0.5]{images/plot4.png}
\end{multicols*}
\end{document}